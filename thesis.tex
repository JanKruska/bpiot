% This is samplepaper.tex, a sample chapter demonstrating the
% LLNCS macro package for Springer Computer Science proceedings;
% Version 2.20 of 2017/10/04
%
\documentclass[runningheads]{template/llncs}
%
\usepackage{graphicx}
\usepackage{todonotes}
% Used for displaying a sample figure. If possible, figure files should
% be included in EPS format.
%
% If you use the hyperref package, please uncomment the following line
% to display URLs in blue roman font according to Springer's eBook style:
% \renewcommand\UrlFont{\color{blue}\rmfamily}

\begin{document}
%
\title{Thesis paper in Seminar Business Processes and the Internet of Things on Conformance Checking Using Activity and Trace Embeddings.}
%
%\titlerunning{Abbreviated paper title}
% If the paper title is too long for the running head, you can set
% an abbreviated paper title here
%
\author{Jan Kruska}
%
%\authorrunning{Pegoraro et al.}
% First names are abbreviated in the running head.
% If there are more than two authors, 'et al.' is used.
%
%\institute{Chair of Process and Data Science\\Department of Computer Science, RWTH Aachen, Aachen, Germany
%	\email{\{pegoraro,wvdaalst\}@pads.rwth-aachen.de}\\
%	\url{http://www.pads.rwth-aachen.de/}}
%
\maketitle              % typeset the header of the contribution
%
\begin{abstract}
\todo{Abstract}
%\keywords{First keyword  \and Second keyword \and Another keyword.}
\end{abstract}
%
%
%
\section{Planned structure}
The following list shall give a short overview of both the intended general structure of the seminar paper. In each section the main questions that should be answered by it are listed.
\begin{enumerate}
	\item Abstract
	\item Introduction
	\begin{enumerate}
		What is the motivation of the paper
	\end{enumerate}
	\item Background
	\begin{enumerate}
		\item What is conformance checking, how is it useful and why would you do it, which methods are there, what are advantages and disadvantages of said methods.
		\item What is NLP (briefly), what are embeddings generally and word embeddings specifically, what are word2vec and doc2vec and how do they approach the NLP problems.
		\item What parallels are there between NLP and conformance checking
		\item If there are these parallels, how could such NLP mechanism be adapted (theoretically not practically) for conformance checking, what could the advantages be of such an approach, what could be disadvantages and/or problems that will need addressing.
	\end{enumerate}
	\item Method
	\begin{enumerate}
		\item How were word2vec and doc2vec adapted to conformance checking in the paper (I.e. practical adaption)
	\end{enumerate}
	\item Results
	\begin{enumerate}
		\item What are the results given in the paper
		\item What metrics were used to evaluate the performance of the approach presented in the paper, are they appropriate, are there other possible metrics, if so why was this one choses, what advantages or disadvantages do different metrics have.
		\item What are the results when using their code and reproducing the results (ideally the same)
	\end{enumerate}
	\item Outlook
	\begin{enumerate}
		\item What future work do the authors suggest
		\item What problems were encountered when trying to reproduce the results from the paper. Could they be reproduced completely, just in part or not at all. If so are there suggestions on how could these problems be mitigated.
	\end{enumerate}
\end{enumerate}

\section{Aims}
The main focus of the seminar paper will be a look at \cite{PBWe20}. The seminar paper should give an overview of the backgrounds that are omitted for brevity in the paper.
Aside from that 
Since the authors provide accompanying source code to their paper it is also possible to try and reproduce their results.
As of now it is unclear if that will be possible within a reasonable time frame, as an actual implementation is outside of the scope of this seminar.
Furthermore the aim of the seminar is to create a connection between business process intelligence and the Internet of Things. From this perspective there are two main interests to be derived: Firstly is it possible to apply their proposed method on limited hardware, as this is generally the case in the Internet of Things.  %especially regarding processing speeds
Secondly and probably more importantly the method in the paper is described in terms of an offline analysis, where process models and a full event log are available. In the Internet of Things however continuos processes are prevalent, and in general we would like to be able to detect deviations (i.e. non-conformance) already while the process is still running.
%
% ---- Bibliography ----
%
% BibTeX users should specify bibliography style 'splncs04'.
% References will then be sorted and formatted in the correct style.
%
\bibliographystyle{template/splncs04}
\bibliography{literature/bibliography}
%
\end{document}
